\documentclass{article}
\usepackage{czt}
\usepackage{circus}

\begin{document}
\begin{zsection}
  \SECTION circus \parents standard\_toolkit
\end{zsection}

\begin{zed}
  [ COOKIE , OutputCookie ]
\end{zed}


\begin{circus}
	\circchannel a: \nat 
\end{circus}

\begin{circus}
   \circprocess\ HRModule \circdef \circbegin
\end{circus}


\begin{zed}
  [ EMP , NAME , ADDRESS , TEL]
\end{zed}

\begin{axdef}
  EMPDETAIL == NAME \cross ADDRESS \cross TEL	\\
  hrstaff : \power EMP							\\
\end{axdef}

%sfsdfdf
\begin{schema}{State}
  empDet : EMP \pfun EMPDETAIL
\end{schema}
\begin{circus}
\circstate\ State ~~==~~ [~ empDet : EMP \pfun EMPDETAIL ~]  \\
    InitState ~~==~~ [~ State~' | money' = 0 \land quantity' = cookieQuantity ~] \\
    InputMoney ~~==~~ [~ \Delta State; x?: \nat | money \leq cookieValue \land money' = money + x? ~] \\
\end{circus}
    
\begin{schema}{InitState}
  State~'
\end{schema}

\begin{schema}{getEmployeeDetail}
  \Xi State					\\
  emp? : EMP				\\
  req? : EMP				\\ 
  detail! : EMPDETAIL		\\
\where
  req? \in \dom empDet	\land (detail! = empDet\ emp?) \\
\end{schema}

\begin{circusaction}
   \circstate\ State
\end{circusaction}

\begin{circusaction}
    InitState \circseq (\circmu\ X \circspot getEmployeeDetail \circseq X)
\end{circusaction}


\begin{circus}
	\circend 
\end{circus}


This specification describes ...



\end{document}