\documentclass{article}
\usepackage{a4}
\usepackage{makeidx}
\usepackage{overture}

\newcommand{\StateDef}[1]{{\bf #1}}
\newcommand{\TypeDef}[1]{{\bf #1}}
\newcommand{\TypeOcc}[1]{{\it #1}}
\newcommand{\FuncDef}[1]{{\bf #1}}
\newcommand{\FuncOcc}[1]{#1}
\newcommand{\ModDef}[1]{{\tiny #1}}

\makeindex

\title{A crosswords assistant}
\author{Yves LEDRU\footnotemark}
\date{March 14, 1995}

\begin{document}
\maketitle
\footnotetext{Laboratoire de G\'enie Informatique - Institut IMAG (UJF - INPG-
CNRS), BP 
53x 38041 Grenoble
Cedex (FRANCE) - Tel + 33 76827214 - Fax + 33 76446675 - e-mail
Yves.Ledru@Imag.fr}


This tutorial example is taken out of a  VDM course given
to the students of the Dipl\^{o}me d'Etudes Sup\'erieures Sp\'ecialis\'ees en
G\'enie Informatique (5th year) at the Universit\'e Joseph Fourier.
This
example uses the implicit style of specification of VDM-SL and thus may not be
executed with the IFAD toolbox.


\section{Informal statement}

The crosswords assistant is a simple system which helps writing crosswords. 
Its user places words or black squares on a a crossword grid. 
The system  helps him keep a log of the words that appear on the grid.
These words  appear in a list of words (waiting list). 
The user will then informally check that these words effectively exist. Once a
word has been checked by the user, it will be validated and stored in a second
list. 

\subsection*{Example}

The user has placed sucessively the words {\tt word}, {\tt coal} and {\tt art}
on the grid. 

\vspace{0.5cm}

{\tt \begin{tabular}{r|c|c|c|c|c|c|c|c|}
 & 1 & 2 & 3 & 4 & 5 & 6 & 7 & 8\\
\hline
1 & & & & & & & & \\
\hline
2 & & & & & & & & \\
\hline
3 & & & &w& & & & \\
\hline
4 & & &c &o &a &l & & \\
\hline
5 & & &a &r&t & & & \\
\hline
6 & & & &d& & & & \\
\hline
7 & & & & & & & & \\
\hline
8 & & & & & & & & \\
\hline
\end{tabular}}

\vspace{0.5cm}

As a result, the words {\tt at} and {\tt ca} are also on the grid. The list of
words to validate is thus:

\begin{itemize}
\item[ ] words to validate : {\tt word, coal, art, at, ca}
\end{itemize}

The user then checks in his (paper) dictionary that all words but {\tt ca} are
english words. The two lists become:

\begin{itemize}
\item[ ] words to validate : {\tt ca}
\item[ ] valid words : {\tt word, coal, art, at}
\end{itemize}

In the sequel, the list of words to validate will also be referred to as the
``waiting list''.

\newpage
The user may now add the word {\tt cord} to the grid.

\vspace{0.5cm}

{\tt \begin{tabular}{r|c|c|c|c|c|c|c|c|}
 & 1 & 2 & 3 & 4 & 5 & 6 & 7 & 8\\
\hline
\multicolumn{9}{l}{\ldots}\\
\hline
3 & & & &w& & & & \\
\hline
4 & & &c &o &a &l & & \\
\hline
5 & & &a &r&t & & & \\
\hline
6 &c &o &r &d& & & & \\
\hline
\multicolumn{9}{l}{\ldots}\\
\end{tabular}}

\vspace{0.5cm}

{\tt ca} has now disappeared from the grid and the waiting list
is updated accordingly.

\begin{itemize}
\item[ ] words to validate : {\tt car, cord}
\item[ ] valid words : {\tt word, coal, art, at}
\end{itemize}

Other operations on the grid include:

\begin{itemize}
\item adding ``black'' squares
\item deleting some letters
\end{itemize}

The objective of the user is to fill in the whole grid with either black
squares or letters and to end up with an empty waiting list. 

\documentclass{article}
\usepackage{a4}
\usepackage{makeidx}
\usepackage{overture}

\newcommand{\StateDef}[1]{{\bf #1}}
\newcommand{\TypeDef}[1]{{\bf #1}}
\newcommand{\TypeOcc}[1]{{\it #1}}
\newcommand{\FuncDef}[1]{{\bf #1}}
\newcommand{\FuncOcc}[1]{#1}
\newcommand{\ModDef}[1]{{\tiny #1}}

\makeindex

\title{A crosswords assistant}
\author{Yves LEDRU\footnotemark}
\date{March 14, 1995}

\begin{document}
\maketitle
\footnotetext{Laboratoire de G\'enie Informatique - Institut IMAG (UJF - INPG-
CNRS), BP 
53x 38041 Grenoble
Cedex (FRANCE) - Tel + 33 76827214 - Fax + 33 76446675 - e-mail
Yves.Ledru@Imag.fr}


This tutorial example is taken out of a  VDM course given
to the students of the Dipl\^{o}me d'Etudes Sup\'erieures Sp\'ecialis\'ees en
G\'enie Informatique (5th year) at the Universit\'e Joseph Fourier.
This
example uses the implicit style of specification of VDM-SL and thus may not be
executed with the IFAD toolbox.


\section{Informal statement}

The crosswords assistant is a simple system which helps writing crosswords. 
Its user places words or black squares on a a crossword grid. 
The system  helps him keep a log of the words that appear on the grid.
These words  appear in a list of words (waiting list). 
The user will then informally check that these words effectively exist. Once a
word has been checked by the user, it will be validated and stored in a second
list. 

\subsection*{Example}

The user has placed sucessively the words {\tt word}, {\tt coal} and {\tt art}
on the grid. 

\vspace{0.5cm}

{\tt \begin{tabular}{r|c|c|c|c|c|c|c|c|}
 & 1 & 2 & 3 & 4 & 5 & 6 & 7 & 8\\
\hline
1 & & & & & & & & \\
\hline
2 & & & & & & & & \\
\hline
3 & & & &w& & & & \\
\hline
4 & & &c &o &a &l & & \\
\hline
5 & & &a &r&t & & & \\
\hline
6 & & & &d& & & & \\
\hline
7 & & & & & & & & \\
\hline
8 & & & & & & & & \\
\hline
\end{tabular}}

\vspace{0.5cm}

As a result, the words {\tt at} and {\tt ca} are also on the grid. The list of
words to validate is thus:

\begin{itemize}
\item[ ] words to validate : {\tt word, coal, art, at, ca}
\end{itemize}

The user then checks in his (paper) dictionary that all words but {\tt ca} are
english words. The two lists become:

\begin{itemize}
\item[ ] words to validate : {\tt ca}
\item[ ] valid words : {\tt word, coal, art, at}
\end{itemize}

In the sequel, the list of words to validate will also be referred to as the
``waiting list''.

\newpage
The user may now add the word {\tt cord} to the grid.

\vspace{0.5cm}

{\tt \begin{tabular}{r|c|c|c|c|c|c|c|c|}
 & 1 & 2 & 3 & 4 & 5 & 6 & 7 & 8\\
\hline
\multicolumn{9}{l}{\ldots}\\
\hline
3 & & & &w& & & & \\
\hline
4 & & &c &o &a &l & & \\
\hline
5 & & &a &r&t & & & \\
\hline
6 &c &o &r &d& & & & \\
\hline
\multicolumn{9}{l}{\ldots}\\
\end{tabular}}

\vspace{0.5cm}

{\tt ca} has now disappeared from the grid and the waiting list
is updated accordingly.

\begin{itemize}
\item[ ] words to validate : {\tt car, cord}
\item[ ] valid words : {\tt word, coal, art, at}
\end{itemize}

Other operations on the grid include:

\begin{itemize}
\item adding ``black'' squares
\item deleting some letters
\end{itemize}

The objective of the user is to fill in the whole grid with either black
squares or letters and to end up with an empty waiting list. 

\documentclass{article}
\usepackage{a4}
\usepackage{makeidx}
\usepackage{overture}

\newcommand{\StateDef}[1]{{\bf #1}}
\newcommand{\TypeDef}[1]{{\bf #1}}
\newcommand{\TypeOcc}[1]{{\it #1}}
\newcommand{\FuncDef}[1]{{\bf #1}}
\newcommand{\FuncOcc}[1]{#1}
\newcommand{\ModDef}[1]{{\tiny #1}}

\makeindex

\title{A crosswords assistant}
\author{Yves LEDRU\footnotemark}
\date{March 14, 1995}

\begin{document}
\maketitle
\footnotetext{Laboratoire de G\'enie Informatique - Institut IMAG (UJF - INPG-
CNRS), BP 
53x 38041 Grenoble
Cedex (FRANCE) - Tel + 33 76827214 - Fax + 33 76446675 - e-mail
Yves.Ledru@Imag.fr}


This tutorial example is taken out of a  VDM course given
to the students of the Dipl\^{o}me d'Etudes Sup\'erieures Sp\'ecialis\'ees en
G\'enie Informatique (5th year) at the Universit\'e Joseph Fourier.
This
example uses the implicit style of specification of VDM-SL and thus may not be
executed with the IFAD toolbox.


\section{Informal statement}

The crosswords assistant is a simple system which helps writing crosswords. 
Its user places words or black squares on a a crossword grid. 
The system  helps him keep a log of the words that appear on the grid.
These words  appear in a list of words (waiting list). 
The user will then informally check that these words effectively exist. Once a
word has been checked by the user, it will be validated and stored in a second
list. 

\subsection*{Example}

The user has placed sucessively the words {\tt word}, {\tt coal} and {\tt art}
on the grid. 

\vspace{0.5cm}

{\tt \begin{tabular}{r|c|c|c|c|c|c|c|c|}
 & 1 & 2 & 3 & 4 & 5 & 6 & 7 & 8\\
\hline
1 & & & & & & & & \\
\hline
2 & & & & & & & & \\
\hline
3 & & & &w& & & & \\
\hline
4 & & &c &o &a &l & & \\
\hline
5 & & &a &r&t & & & \\
\hline
6 & & & &d& & & & \\
\hline
7 & & & & & & & & \\
\hline
8 & & & & & & & & \\
\hline
\end{tabular}}

\vspace{0.5cm}

As a result, the words {\tt at} and {\tt ca} are also on the grid. The list of
words to validate is thus:

\begin{itemize}
\item[ ] words to validate : {\tt word, coal, art, at, ca}
\end{itemize}

The user then checks in his (paper) dictionary that all words but {\tt ca} are
english words. The two lists become:

\begin{itemize}
\item[ ] words to validate : {\tt ca}
\item[ ] valid words : {\tt word, coal, art, at}
\end{itemize}

In the sequel, the list of words to validate will also be referred to as the
``waiting list''.

\newpage
The user may now add the word {\tt cord} to the grid.

\vspace{0.5cm}

{\tt \begin{tabular}{r|c|c|c|c|c|c|c|c|}
 & 1 & 2 & 3 & 4 & 5 & 6 & 7 & 8\\
\hline
\multicolumn{9}{l}{\ldots}\\
\hline
3 & & & &w& & & & \\
\hline
4 & & &c &o &a &l & & \\
\hline
5 & & &a &r&t & & & \\
\hline
6 &c &o &r &d& & & & \\
\hline
\multicolumn{9}{l}{\ldots}\\
\end{tabular}}

\vspace{0.5cm}

{\tt ca} has now disappeared from the grid and the waiting list
is updated accordingly.

\begin{itemize}
\item[ ] words to validate : {\tt car, cord}
\item[ ] valid words : {\tt word, coal, art, at}
\end{itemize}

Other operations on the grid include:

\begin{itemize}
\item adding ``black'' squares
\item deleting some letters
\end{itemize}

The objective of the user is to fill in the whole grid with either black
squares or letters and to end up with an empty waiting list. 

\documentclass{article}
\usepackage{a4}
\usepackage{makeidx}
\usepackage{overture}

\newcommand{\StateDef}[1]{{\bf #1}}
\newcommand{\TypeDef}[1]{{\bf #1}}
\newcommand{\TypeOcc}[1]{{\it #1}}
\newcommand{\FuncDef}[1]{{\bf #1}}
\newcommand{\FuncOcc}[1]{#1}
\newcommand{\ModDef}[1]{{\tiny #1}}

\makeindex

\title{A crosswords assistant}
\author{Yves LEDRU\footnotemark}
\date{March 14, 1995}

\begin{document}
\maketitle
\footnotetext{Laboratoire de G\'enie Informatique - Institut IMAG (UJF - INPG-
CNRS), BP 
53x 38041 Grenoble
Cedex (FRANCE) - Tel + 33 76827214 - Fax + 33 76446675 - e-mail
Yves.Ledru@Imag.fr}


This tutorial example is taken out of a  VDM course given
to the students of the Dipl\^{o}me d'Etudes Sup\'erieures Sp\'ecialis\'ees en
G\'enie Informatique (5th year) at the Universit\'e Joseph Fourier.
This
example uses the implicit style of specification of VDM-SL and thus may not be
executed with the IFAD toolbox.


\section{Informal statement}

The crosswords assistant is a simple system which helps writing crosswords. 
Its user places words or black squares on a a crossword grid. 
The system  helps him keep a log of the words that appear on the grid.
These words  appear in a list of words (waiting list). 
The user will then informally check that these words effectively exist. Once a
word has been checked by the user, it will be validated and stored in a second
list. 

\subsection*{Example}

The user has placed sucessively the words {\tt word}, {\tt coal} and {\tt art}
on the grid. 

\vspace{0.5cm}

{\tt \begin{tabular}{r|c|c|c|c|c|c|c|c|}
 & 1 & 2 & 3 & 4 & 5 & 6 & 7 & 8\\
\hline
1 & & & & & & & & \\
\hline
2 & & & & & & & & \\
\hline
3 & & & &w& & & & \\
\hline
4 & & &c &o &a &l & & \\
\hline
5 & & &a &r&t & & & \\
\hline
6 & & & &d& & & & \\
\hline
7 & & & & & & & & \\
\hline
8 & & & & & & & & \\
\hline
\end{tabular}}

\vspace{0.5cm}

As a result, the words {\tt at} and {\tt ca} are also on the grid. The list of
words to validate is thus:

\begin{itemize}
\item[ ] words to validate : {\tt word, coal, art, at, ca}
\end{itemize}

The user then checks in his (paper) dictionary that all words but {\tt ca} are
english words. The two lists become:

\begin{itemize}
\item[ ] words to validate : {\tt ca}
\item[ ] valid words : {\tt word, coal, art, at}
\end{itemize}

In the sequel, the list of words to validate will also be referred to as the
``waiting list''.

\newpage
The user may now add the word {\tt cord} to the grid.

\vspace{0.5cm}

{\tt \begin{tabular}{r|c|c|c|c|c|c|c|c|}
 & 1 & 2 & 3 & 4 & 5 & 6 & 7 & 8\\
\hline
\multicolumn{9}{l}{\ldots}\\
\hline
3 & & & &w& & & & \\
\hline
4 & & &c &o &a &l & & \\
\hline
5 & & &a &r&t & & & \\
\hline
6 &c &o &r &d& & & & \\
\hline
\multicolumn{9}{l}{\ldots}\\
\end{tabular}}

\vspace{0.5cm}

{\tt ca} has now disappeared from the grid and the waiting list
is updated accordingly.

\begin{itemize}
\item[ ] words to validate : {\tt car, cord}
\item[ ] valid words : {\tt word, coal, art, at}
\end{itemize}

Other operations on the grid include:

\begin{itemize}
\item adding ``black'' squares
\item deleting some letters
\end{itemize}

The objective of the user is to fill in the whole grid with either black
squares or letters and to end up with an empty waiting list. 

\include{generated/latex/specification/crossword.vdmsl}

\end{document}


\end{document}


\end{document}


\end{document}
